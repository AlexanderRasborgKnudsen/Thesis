\chapter{Introduction}
When acquiring data in nuclear physic, a high spacial and time resolution is necessary to investigates the nuclear process. 
The high resolution results in a immense analog data, flowing from the detectors. 
To process the nuclear spectrometry signal involves finding the energy and timestamps of the particle detected.
A major problem is particles hitting the detector in a short period of time, resulting in a pile up and discarding the measurements.\newline
Therefore, in this thesis I investigate and propose a method for processing the nuclear spectrometry signals based upon a Field Programmable Gate Array (FPGA) solution.
The

The thesis is conducted in cooperation with the department of Physics of Aarhus University.
With their current system are analog, which have multiple issues such as discarded pulses, inflexibility and 
Therefore, this thesis seeks to process multiple channels real-time and outputs a energy histogram of the detected particles by using a new proposed method.

Test\cite{LatexModulesLink} \cite{dummy2015}\fxnote{test af biblatex}
\section{Reading Guide}
The structure of the Thesis is shown below:
\subparagraph{Chapter 1} The introduction. 
\subparagraph{Chapter 2} The background. 

\section{Problem Definition}

\section{Thesis goal, Approach and Scope}
This section describes the goals, the approach used and the scope of the master thesis.
\subsection{Thesis goal}

\paragraph{Goal 1:}

\paragraph{Goal 2:}

\paragraph{Goal 3:}

\subsection{Approach}
 
\paragraph{Phase 1: Preliminary work and background}
 The first phase is carried out prior to the thesis. 
\paragraph{Phase 2: Experiments}
 The second phase involves the conduction of a number of experiments. The purpose of this phase is to narrow down the proposed methods in the first phase to only the most promising one.
\paragraph{Phase 3 : Analysis}
 In the third phase the proposed method is investigated further and research activities are
 carried out in order to study and assess related algorithms in other application fields.
\paragraph{Phase 4: First Design}
 The fourth phase deals with the development of a first algorithm. The phase
 builds upon the concepts explored throughout the analysis.
\paragraph{Phase 5: Advanced Design}
 The fifth phase involves two subgoals improving both timing performance and accuracy. A
 real-time system is designed executing the proposed algorithm according to the specified
 performance requirements. Based on the results of the first proposed algorithm, possible
 improvements are investigated in order to solve potential problems.
\paragraph{Phase 6: Evaluation}
 In the last phase the proposed algorithms are evaluated and compared based on an acquired
 test set. Accuracy and timing performance are recorded and compared to the real-time
 requirements.
 
 \subsection{Scope}
 